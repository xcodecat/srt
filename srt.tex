\documentclass{report}
\usepackage{tcolorbox}
\usepackage{amssymb}
\usepackage{enumitem}
\usepackage{fancyvrb}
\usepackage{sectsty}
\usepackage{siunitx}
\usepackage{bm}

% Set section titles to bold and italic
\allsectionsfont{\normalfont\bfseries\itshape}

\begin{document}
\begin{titlepage}
    \centering
	{\LARGE \textsc{Spezielle Relativitätstheorie}\par}
	\vspace{1.5cm}
	{\huge\bfseries Physik\par}
	\vspace{2cm}
	{\Large\itshape Laura Thiel\par}

% Bottom of the page
	{\large \today\par}
\end{titlepage}
\tableofcontents
\chapter{Einstieg SRT}
\section{Das Jahr 1905}
Im Jahr 1905 erlebte die Welt der Wissenschaft das faszinierende \textbf{"Wunderjahr"} von Albert Einstein, in dem er mit einer beeindruckenden Vielfalt an bahnbrechenden Arbeiten die Grundlagen für das Verständnis des Universums legte. Seine Veröffentlichungen in diesem Jahr sind wie ein kreativer Kosmos, der die Grenzen des Denkens erweiterte.
Einstein präsentierte nicht nur die \textbf{spezielle Relativitätstheorie}, die unsere Konzeption von Raum und Zeit revolutionierte, sondern wagte sich auch an die Hypothese der \textbf{Lichtquanten}, die die Wellen-Teilchen-Dualität des Lichts in den Fokus rückte. Sein Genie zeigte sich ebenfalls in Arbeiten zur \textbf{Brownschen Molekularbewegung} und der \textbf{Erklärung des photoelektrischen Effekts}, die die Quantenmechanik vorantrieben.
Diese wegweisenden Veröffentlichungen katapultierten Einstein in den wissenschaftlichen Olymp und ebneten den Weg für eine glanzvolle Karriere. Seine Doktorarbeit wurde 1905 in Zürich angenommen, und im darauf folgenden Jahr erhielt er den Titel "Dr. Albert Einstein". Die kontinuierliche Anerkennung führte zu seiner Beförderung in seinem Berufsfeld und kulminierte schließlich 1909 in einer Professur für Physik in Zürich.
Das Jahr 1905 markiert nicht nur den Höhepunkt von Einsteins kreativem Schaffen, sondern auch den Beginn eines erstaunlichen wissenschaftlichen Erbes, das die Welt bis heute prägt. Einsteins \textbf{Wunderjahr} bleibt eine inspirierende Quelle für alle, die die Mysterien des Universums erforschen und die Grenzen des menschlichen Denkens erweitern wollen.
\section{Einsteins Gedankenwelt}
Einsteins Gedankenwelt offenbarte sich in seiner Speziellen Relativitätstheorie, die im Volksmund gerne als \emph{"Alles ist relativ"} zusammengefasst wird. Doch der wahre Durchbruch Einsteins bestand darin, eine absolute Größe festzusetzen und so die Grundlagen der Physik zu revolutionieren.
\subsection{(Klein-)Einstein rennt dem Licht hinterher}
Einstein beschäftigte sich bereits in seiner Jugend mit Gedankenexperimenten, darunter die Vorstellung, einem Lichtstrahl hinterherzulaufen. Diese Experimente führten ihn zu bedeutenden Erkenntnissen über die Natur der Zeit und des Raums.
\subsection{Michelson und Morley Experiment}
Das berühmte Michelson-Morley-Experiment markierte einen Meilenstein in der Geschichte der Physik. Die Forscher maßen die Lichtgeschwindigkeit parallel und antiparallel zur Bahngeschwindigkeit der Erde. Ein zentrales Gedankenexperiment Einsteins während dieser Zeit war die Frage, ob er sich selbst sehen könnte, wenn er das Licht einholen würde, und was er überhaupt sehen könnte, wenn er schneller als das Licht wäre.
\subsection{Verbesserung der Messapparaturen}
In der Ära von Michelson und Morley wurden stetig verbesserte Messapparaturen entwickelt, um die Lichtgeschwindigkeit zu präzisieren. Durch den geschickten Einsatz der Erdbahngeschwindigkeit um die Sonne als Maßstab erwartete man Veränderungen in der Lichtgeschwindigkeit – ähnlich wie bei einem Boot, das konstant flussaufwärts und flussabwärts angetrieben wird.
\subsection{Ergebnis des Michelson-Morley Experiments}
\sloppy % <-- Add this line
Überraschenderweise ergab das Experiment jedoch keine Änderung in der Lichtgeschwindigkeit.
Die Physiker kamen zu dem Ergebnis, dass die Lichtgeschwindigkeit einen konstanten Wert hat: 
\begin{center}
    \protect\boldmath $c = \SI{299792458}{\meter\per\second} \approx \SI{300000}{\kilo\meter\per\second}$\unboldmath.
\end{center}
Dies bedeutet, dass ein Einstein, der gleichförmig rennt, den gleichen Wert für die Lichtgeschwindigkeit misst wie ein ruhender Einstein. 
In diesem Kontext ist also nichts relativ; die Lichtgeschwindigkeit ist schlichtweg eine absolute Größe.

\end{document}

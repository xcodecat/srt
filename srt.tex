\documentclass{report}
\usepackage[ngerman]{babel}
\usepackage{tcolorbox}
\usepackage{amssymb}
\usepackage{enumitem}
\usepackage{fancyvrb}
\usepackage{sectsty}
\usepackage{siunitx}
\usepackage{bm}

% Set section titles to bold and italic
\allsectionsfont{\normalfont\bfseries\itshape}

\begin{document}
\begin{titlepage}
    \centering
	{\LARGE \textsc{Spezielle Relativitätstheorie}\par}
	\vspace{1.5cm}
	{\huge\bfseries Physik\par}
	\vspace{2cm}
	{\Large\itshape Laura Thiel\par}
	{\large \today\par}
\end{titlepage}
\tableofcontents
\chapter{Einstieg SRT}
\section{Das Jahr 1905}
Im Jahr 1905 erlebte die Welt der Wissenschaft das faszinierende \textbf{"Wunderjahr"} von Albert Einstein, in dem er mit einer beeindruckenden Vielfalt an bahnbrechenden Arbeiten die Grundlagen für das Verständnis des Universums legte. Seine Veröffentlichungen in diesem Jahr sind wie ein kreativer Kosmos, der die Grenzen des Denkens erweiterte.
Einstein präsentierte nicht nur die \textbf{spezielle Relativitätstheorie}, die unsere Konzeption von Raum und Zeit revolutionierte, sondern wagte sich auch an die Hypothese der \textbf{Lichtquanten}, die die Wellen-Teilchen-Dualität des Lichts in den Fokus rückte. Sein Genie zeigte sich ebenfalls in Arbeiten zur \textbf{Brownschen Molekularbewegung} und der \textbf{Erklärung des photoelektrischen Effekts}, die die Quantenmechanik vorantrieben.
Diese wegweisenden Veröffentlichungen katapultierten Einstein in den wissenschaftlichen Olymp und ebneten den Weg für eine glanzvolle Karriere. Seine Doktorarbeit wurde 1905 in Zürich angenommen, und im darauf folgenden Jahr erhielt er den Titel "Dr. Albert Einstein". Die kontinuierliche Anerkennung führte zu seiner Beförderung in seinem Berufsfeld und kulminierte schließlich 1909 in einer Professur für Physik in Zürich.
Das Jahr 1905 markiert nicht nur den Höhepunkt von Einsteins kreativem Schaffen, sondern auch den Beginn eines erstaunlichen wissenschaftlichen Erbes, das die Welt bis heute prägt. Einsteins \textbf{Wunderjahr} bleibt eine inspirierende Quelle für alle, die die Mysterien des Universums erforschen und die Grenzen des menschlichen Denkens erweitern wollen.
\section{Einsteins Gedankenwelt}
Einsteins Gedankenwelt offenbarte sich in seiner Speziellen Relativitätstheorie, die im Volksmund gerne als \emph{"Alles ist relativ"} zusammengefasst wird. Doch der wahre Durchbruch Einsteins bestand darin, eine absolute Größe festzusetzen und so die Grundlagen der Physik zu revolutionieren.
\subsection{(Klein-)Einstein rennt dem Licht hinterher}
Einstein beschäftigte sich bereits in seiner Jugend mit Gedankenexperimenten, darunter die Vorstellung, einem Lichtstrahl hinterherzulaufen. Diese Experimente führten ihn zu bedeutenden Erkenntnissen über die Natur der Zeit und des Raums.
\subsection{Michelson und Morley Experiment}
Das berühmte Michelson-Morley-Experiment markierte einen Meilenstein in der Geschichte der Physik. Die Forscher maßen die Lichtgeschwindigkeit parallel und antiparallel zur Bahngeschwindigkeit der Erde. Ein zentrales Gedankenexperiment Einsteins während dieser Zeit war die Frage, ob er sich selbst sehen könnte, wenn er das Licht einholen würde, und was er überhaupt sehen könnte, wenn er schneller als das Licht wäre.
\subsection{Verbesserung der Messapparaturen}
In der Ära von Michelson und Morley wurden stetig verbesserte Messapparaturen entwickelt, um die Lichtgeschwindigkeit zu präzisieren. Durch den geschickten Einsatz der Erdbahngeschwindigkeit um die Sonne als Maßstab erwartete man Veränderungen in der Lichtgeschwindigkeit – ähnlich wie bei einem Boot, das konstant flussaufwärts und flussabwärts angetrieben wird.
\subsection{Ergebnis des Michelson-Morley Experiments}
\sloppy % <-- Add this line
Überraschenderweise ergab das Experiment jedoch keine Änderung in der Lichtgeschwindigkeit.
Die Physiker kamen zu dem Ergebnis, dass die Lichtgeschwindigkeit einen konstanten Wert hat: 
$$c = \SI{299792458}{\meter\per\second} \approx \SI{300000}{\kilo\meter\per\second}$$
Dies bedeutet, dass ein Einstein, der gleichförmig rennt, den gleichen Wert für die Lichtgeschwindigkeit misst wie ein ruhender Einstein. 
In diesem Kontext ist also nichts relativ; die Lichtgeschwindigkeit ist schlichtweg eine absolute Größe.
\chapter{Einstein'sche Postulate und Bezugssysteme}
% \emph{Postulieren, ursprünglich im 15. Jahrhundert als 'unbedingt fordern', später in der Philosophie des 18. Jahrhunderts als 'ein vorläufiges Akzeptieren als wahr' verwendet, stammt vom lateinischen "postulāre" ab, was 'verlangen, begehren, erwarten, fordern' bedeutet. In der Philosophie bezieht sich ein Postulat auf eine unbedingte Forderung oder eine vorläufig angenommene These, die möglicherweise noch nicht bewiesen werden kann. Der Begriff "Postulat" selbst stammt vom lateinischen "postulātum" ab, was 'Forderung, gerichtlicher Antrag, Gesuch' bedeutet.}

Einstein legte lediglich zwei grundlegende Annahmen oder Postulate fest, um seine wegweisende Theorie der Speziellen Relativitätstheorie zu entwickeln. Bereits kennengelernt haben wir die Konstanz der Lichtgeschwindigkeit (c), die experimentell nie widerlegt wurde und ein fundamentaler Bestandteil der Theorie ist. 
Im nächsten Abschnitt werden wir uns mit dem zweiten Postulat beschäftigen, dem sogenannten Relativitätsprinzip.
\section{Bezugssysteme und Inertialsysteme}
Das mechanische \textbf{Relativitätsprinzip} geht schon auf \textbf{Galilei} und \textbf{Newton} zurück. In den folgenden Unterkapiteln geht es nun um das 'Relative' in der Speziellen Relativitätstheorie (ab hier SRT genannt).
\subsection{Situation 1}
Einstein und Newton bewegen sich in ihren Zügen mit jeweils einer Geschwindigkeit von \(v_{\text{Newton}} = v_{\text{Einstein}} = 47 \, \text{km/h}\). Dadurch bewegen sich die beiden relativ zueinander mit einer Geschwindigkeit von \(v_{\text{relativ}} = 0 \, \text{km/h}\).
Angenommen, Newton führt auf seinem gleichförmig bewegten Zug einen Versuch durch, beispielsweise das Fallenlassen eines Apfels, so würde Einstein dies genauso wahrnehmen wie Newton. Diese Situation wäre gleich, selbst wenn beide Wissenschaftler in einem Labor stehen würden.
\subsection{Situation 2}
Im zweiten Szenario würde Einstein in seinem Zug eine Relativgeschwindigkeit von \(v_{\text{relativ}} = 65 \, \text{km/h} - 47 \, \text{km/h} = 18 \, \text{km/h}\) in Bezug auf Newton messen. Wenn Einstein nun einen Apfel fallen lässt, würde dieser aus seiner Perspektive senkrecht nach unten fallen, während Newton eine Wurfparabel sehen würde, da sich der Apfel aus seiner Perspektive zusätzlich mit \(v_{\text{relativ}}\) nach vorne bewegt.
Wenn dieser Versuch von Einstein in einem ruhenden Labor durchgeführt wird, müsste Newton mit \(v_{\text{relativ}} = 18 \, \text{km/h}\) daran vorbeijoggen, um dieselbe Beobachtung zu machen.
\subsection{Relativitätsprinzip}
Das führt uns zum zweiten Postulat, dem Relativitätsprinzip. In unbeschleunigten Bezugssystemen sollen alle physikalischen Gesetze gleich sein. Galileo Galilei erkannte dies bereits 1638 im Inneren eines ruhigen Schiffes.
In beschleunigten Bezugssystemen variieren die Gesetze aufgrund von Scheinkräften, wie beim Anfahren eines Autos. Im Gegensatz dazu fallen Objekte in unbeschleunigten Bezugssystemen, wie einem gleichförmig fahrenden Zug, senkrecht nach unten. Solche Systeme werden als Inertialsysteme bezeichnet, in denen die physikalischen Gesetze einheitlich gelten, ohne einen absoluten Ruhepunkt.
\section{Lichtuhr}
\end{document}
